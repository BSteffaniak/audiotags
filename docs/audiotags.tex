% Options for packages loaded elsewhere
\PassOptionsToPackage{unicode}{hyperref}
\PassOptionsToPackage{hyphens}{url}
%
\documentclass[
]{book}
\usepackage{lmodern}
\usepackage{amssymb,amsmath}
\usepackage{ifxetex,ifluatex}
\ifnum 0\ifxetex 1\fi\ifluatex 1\fi=0 % if pdftex
  \usepackage[T1]{fontenc}
  \usepackage[utf8]{inputenc}
  \usepackage{textcomp} % provide euro and other symbols
\else % if luatex or xetex
  \usepackage{unicode-math}
  \defaultfontfeatures{Scale=MatchLowercase}
  \defaultfontfeatures[\rmfamily]{Ligatures=TeX,Scale=1}
\fi
% Use upquote if available, for straight quotes in verbatim environments
\IfFileExists{upquote.sty}{\usepackage{upquote}}{}
\IfFileExists{microtype.sty}{% use microtype if available
  \usepackage[]{microtype}
  \UseMicrotypeSet[protrusion]{basicmath} % disable protrusion for tt fonts
}{}
\makeatletter
\@ifundefined{KOMAClassName}{% if non-KOMA class
  \IfFileExists{parskip.sty}{%
    \usepackage{parskip}
  }{% else
    \setlength{\parindent}{0pt}
    \setlength{\parskip}{6pt plus 2pt minus 1pt}}
}{% if KOMA class
  \KOMAoptions{parskip=half}}
\makeatother
\usepackage{xcolor}
\IfFileExists{xurl.sty}{\usepackage{xurl}}{} % add URL line breaks if available
\IfFileExists{bookmark.sty}{\usepackage{bookmark}}{\usepackage{hyperref}}
\hypersetup{
  pdftitle={audiotags Manual},
  pdfauthor={Tianyi Shi},
  hidelinks,
  pdfcreator={LaTeX via pandoc}}
\urlstyle{same} % disable monospaced font for URLs
\usepackage{color}
\usepackage{fancyvrb}
\newcommand{\VerbBar}{|}
\newcommand{\VERB}{\Verb[commandchars=\\\{\}]}
\DefineVerbatimEnvironment{Highlighting}{Verbatim}{commandchars=\\\{\}}
% Add ',fontsize=\small' for more characters per line
\usepackage{framed}
\definecolor{shadecolor}{RGB}{248,248,248}
\newenvironment{Shaded}{\begin{snugshade}}{\end{snugshade}}
\newcommand{\AlertTok}[1]{\textcolor[rgb]{0.94,0.16,0.16}{#1}}
\newcommand{\AnnotationTok}[1]{\textcolor[rgb]{0.56,0.35,0.01}{\textbf{\textit{#1}}}}
\newcommand{\AttributeTok}[1]{\textcolor[rgb]{0.77,0.63,0.00}{#1}}
\newcommand{\BaseNTok}[1]{\textcolor[rgb]{0.00,0.00,0.81}{#1}}
\newcommand{\BuiltInTok}[1]{#1}
\newcommand{\CharTok}[1]{\textcolor[rgb]{0.31,0.60,0.02}{#1}}
\newcommand{\CommentTok}[1]{\textcolor[rgb]{0.56,0.35,0.01}{\textit{#1}}}
\newcommand{\CommentVarTok}[1]{\textcolor[rgb]{0.56,0.35,0.01}{\textbf{\textit{#1}}}}
\newcommand{\ConstantTok}[1]{\textcolor[rgb]{0.00,0.00,0.00}{#1}}
\newcommand{\ControlFlowTok}[1]{\textcolor[rgb]{0.13,0.29,0.53}{\textbf{#1}}}
\newcommand{\DataTypeTok}[1]{\textcolor[rgb]{0.13,0.29,0.53}{#1}}
\newcommand{\DecValTok}[1]{\textcolor[rgb]{0.00,0.00,0.81}{#1}}
\newcommand{\DocumentationTok}[1]{\textcolor[rgb]{0.56,0.35,0.01}{\textbf{\textit{#1}}}}
\newcommand{\ErrorTok}[1]{\textcolor[rgb]{0.64,0.00,0.00}{\textbf{#1}}}
\newcommand{\ExtensionTok}[1]{#1}
\newcommand{\FloatTok}[1]{\textcolor[rgb]{0.00,0.00,0.81}{#1}}
\newcommand{\FunctionTok}[1]{\textcolor[rgb]{0.00,0.00,0.00}{#1}}
\newcommand{\ImportTok}[1]{#1}
\newcommand{\InformationTok}[1]{\textcolor[rgb]{0.56,0.35,0.01}{\textbf{\textit{#1}}}}
\newcommand{\KeywordTok}[1]{\textcolor[rgb]{0.13,0.29,0.53}{\textbf{#1}}}
\newcommand{\NormalTok}[1]{#1}
\newcommand{\OperatorTok}[1]{\textcolor[rgb]{0.81,0.36,0.00}{\textbf{#1}}}
\newcommand{\OtherTok}[1]{\textcolor[rgb]{0.56,0.35,0.01}{#1}}
\newcommand{\PreprocessorTok}[1]{\textcolor[rgb]{0.56,0.35,0.01}{\textit{#1}}}
\newcommand{\RegionMarkerTok}[1]{#1}
\newcommand{\SpecialCharTok}[1]{\textcolor[rgb]{0.00,0.00,0.00}{#1}}
\newcommand{\SpecialStringTok}[1]{\textcolor[rgb]{0.31,0.60,0.02}{#1}}
\newcommand{\StringTok}[1]{\textcolor[rgb]{0.31,0.60,0.02}{#1}}
\newcommand{\VariableTok}[1]{\textcolor[rgb]{0.00,0.00,0.00}{#1}}
\newcommand{\VerbatimStringTok}[1]{\textcolor[rgb]{0.31,0.60,0.02}{#1}}
\newcommand{\WarningTok}[1]{\textcolor[rgb]{0.56,0.35,0.01}{\textbf{\textit{#1}}}}
\usepackage{longtable,booktabs}
% Correct order of tables after \paragraph or \subparagraph
\usepackage{etoolbox}
\makeatletter
\patchcmd\longtable{\par}{\if@noskipsec\mbox{}\fi\par}{}{}
\makeatother
% Allow footnotes in longtable head/foot
\IfFileExists{footnotehyper.sty}{\usepackage{footnotehyper}}{\usepackage{footnote}}
\makesavenoteenv{longtable}
\usepackage{graphicx}
\makeatletter
\def\maxwidth{\ifdim\Gin@nat@width>\linewidth\linewidth\else\Gin@nat@width\fi}
\def\maxheight{\ifdim\Gin@nat@height>\textheight\textheight\else\Gin@nat@height\fi}
\makeatother
% Scale images if necessary, so that they will not overflow the page
% margins by default, and it is still possible to overwrite the defaults
% using explicit options in \includegraphics[width, height, ...]{}
\setkeys{Gin}{width=\maxwidth,height=\maxheight,keepaspectratio}
% Set default figure placement to htbp
\makeatletter
\def\fps@figure{htbp}
\makeatother
\setlength{\emergencystretch}{3em} % prevent overfull lines
\providecommand{\tightlist}{%
  \setlength{\itemsep}{0pt}\setlength{\parskip}{0pt}}
\setcounter{secnumdepth}{5}
\usepackage{booktabs}
\ifluatex
  \usepackage{selnolig}  % disable illegal ligatures
\fi
\usepackage[]{natbib}
\bibliographystyle{apalike}

\title{audiotags Manual}
\author{Tianyi Shi}
\date{2020-10-27}

\begin{document}
\maketitle

{
\setcounter{tocdepth}{1}
\tableofcontents
}
\hypertarget{preface}{%
\chapter*{Preface}\label{preface}}
\addcontentsline{toc}{chapter}{Preface}

Thank you for considering \textbf{audiotags}!

\hypertarget{examples-in-this-manual}{%
\subsection*{Examples in this manual}\label{examples-in-this-manual}}
\addcontentsline{toc}{subsection}{Examples in this manual}

If you want to run the examples in this book:

\begin{enumerate}
\def\labelenumi{\arabic{enumi}.}
\tightlist
\item
  clone the \href{https://github.com/TianyiShi2001/audiotags}{repo} and navigate into it
\item
  create \texttt{src/main.rs}
\item
  all examples, unless otherwise specified, can be copied verbatim from this book to \texttt{src/main.rs} and run with \texttt{cargo\ run} (if you're reading it online, the copy button wil show if you hover over a code block)
\end{enumerate}

\hypertarget{start-simple}{%
\chapter{Start Simple}\label{start-simple}}

The following example shows how you can read an audio file, parse, set, and save its metadata:

\begin{Shaded}
\begin{Highlighting}[]
\KeywordTok{use} \PreprocessorTok{audiotags::}\OperatorTok{\{}\NormalTok{MimeType}\OperatorTok{,}\NormalTok{ Picture}\OperatorTok{,}\NormalTok{ Tag}\OperatorTok{,}\NormalTok{ TagType}\OperatorTok{\};}

\KeywordTok{const}\NormalTok{ MP3\_FILE}\OperatorTok{:} \OperatorTok{\&}\OtherTok{\textquotesingle{}static} \DataTypeTok{str} \OperatorTok{=} \StringTok{"assets/a.mp3"}\OperatorTok{;}

\KeywordTok{fn}\NormalTok{ main() }\OperatorTok{\{}
    \CommentTok{// using \textasciigrave{}default()\textasciigrave{} so that the metadata format is guessed}
    \CommentTok{// (from the file extension) (in this case, Id3v2 tag is read)}
    \KeywordTok{let} \KeywordTok{mut}\NormalTok{ tag }\OperatorTok{=} \PreprocessorTok{Tag::}\KeywordTok{default}\NormalTok{()}\OperatorTok{.}\NormalTok{read\_from\_path(MP3\_FILE)}\OperatorTok{.}\NormalTok{unwrap()}\OperatorTok{;}
    \CommentTok{// You can also specify the metadata format (tag type):}
    \KeywordTok{let}\NormalTok{ \_tag }\OperatorTok{=} \PreprocessorTok{Tag::}\NormalTok{with\_tag\_type(}\PreprocessorTok{TagType::}\NormalTok{Id3v2)}
        \OperatorTok{.}\NormalTok{read\_from\_path(MP3\_FILE)}
        \OperatorTok{.}\NormalTok{expect(}\StringTok{"Fail to read!"}\NormalTok{)}\OperatorTok{;}

\NormalTok{    tag}\OperatorTok{.}\NormalTok{set\_title(}\StringTok{"foo title"}\NormalTok{)}\OperatorTok{;}
    \PreprocessorTok{assert\_eq!}\NormalTok{(tag}\OperatorTok{.}\NormalTok{title()}\OperatorTok{,} \ConstantTok{Some}\NormalTok{(}\StringTok{"foo title"}\NormalTok{))}\OperatorTok{;}
\NormalTok{    tag}\OperatorTok{.}\NormalTok{remove\_title()}\OperatorTok{;}
    \PreprocessorTok{assert!}\NormalTok{(tag}\OperatorTok{.}\NormalTok{title()}\OperatorTok{.}\NormalTok{is\_none())}\OperatorTok{;}
\NormalTok{    tag}\OperatorTok{.}\NormalTok{remove\_title()}\OperatorTok{;}
    \CommentTok{// trying to remove a field that\textquotesingle{}s already empty won\textquotesingle{}t hurt}

    \KeywordTok{let}\NormalTok{ cover }\OperatorTok{=}\NormalTok{ Picture }\OperatorTok{\{}
\NormalTok{        mime\_type}\OperatorTok{:} \PreprocessorTok{MimeType::}\NormalTok{Jpeg}\OperatorTok{,}
\NormalTok{        data}\OperatorTok{:} \OperatorTok{\&}\PreprocessorTok{vec!}\NormalTok{[}\DecValTok{0u8}\OperatorTok{;} \DecValTok{10}\NormalTok{]}\OperatorTok{,}
    \OperatorTok{\};}

\NormalTok{    tag}\OperatorTok{.}\NormalTok{set\_album\_cover(cover}\OperatorTok{.}\NormalTok{clone())}\OperatorTok{;}
    \PreprocessorTok{assert\_eq!}\NormalTok{(tag}\OperatorTok{.}\NormalTok{album\_cover()}\OperatorTok{,} \ConstantTok{Some}\NormalTok{(cover))}\OperatorTok{;}
\NormalTok{    tag}\OperatorTok{.}\NormalTok{remove\_album\_cover()}\OperatorTok{;}
    \PreprocessorTok{assert!}\NormalTok{(tag}\OperatorTok{.}\NormalTok{album\_cover()}\OperatorTok{.}\NormalTok{is\_none())}\OperatorTok{;}
\NormalTok{    tag}\OperatorTok{.}\NormalTok{remove\_album\_cover()}\OperatorTok{;}

\NormalTok{    tag}\OperatorTok{.}\NormalTok{save\_to\_path(MP3\_FILE)}\OperatorTok{.}\NormalTok{expect(}\StringTok{"Fail to save"}\NormalTok{)}\OperatorTok{;}
    \CommentTok{// }\AlertTok{TASK}\CommentTok{: reload the file and prove the data have been saved}
\OperatorTok{\}}
\end{Highlighting}
\end{Shaded}

\hypertarget{conversion}{%
\chapter{Conversion}\label{conversion}}

The following example shows how you can read the tag in an \texttt{mp3} file, convert it into an \texttt{mp4} tag, and write it to an \texttt{m4a} file.

\begin{Shaded}
\begin{Highlighting}[]
\KeywordTok{use} \PreprocessorTok{audiotags::}\OperatorTok{\{}\NormalTok{Config}\OperatorTok{,}\NormalTok{ Tag}\OperatorTok{,}\NormalTok{ TagType}\OperatorTok{\};}

\KeywordTok{fn}\NormalTok{ main() }\OperatorTok{\{}
    \CommentTok{// we have an mp3 and an m4a file}
    \KeywordTok{const}\NormalTok{ MP3\_FILE}\OperatorTok{:} \OperatorTok{\&}\OtherTok{\textquotesingle{}static} \DataTypeTok{str} \OperatorTok{=} \StringTok{"assets/a.mp3"}\OperatorTok{;}
    \KeywordTok{const}\NormalTok{ M4A\_FILE}\OperatorTok{:} \OperatorTok{\&}\OtherTok{\textquotesingle{}static} \DataTypeTok{str} \OperatorTok{=} \StringTok{"assets/a.m4a"}\OperatorTok{;}
    \CommentTok{// read tag from the mp3 file. Using \textasciigrave{}default()\textasciigrave{} so that the}
    \CommentTok{// type of tag is guessed from the file extension}
    \KeywordTok{let} \KeywordTok{mut}\NormalTok{ mp3tag }\OperatorTok{=} \PreprocessorTok{Tag::}\KeywordTok{default}\NormalTok{()}\OperatorTok{.}\NormalTok{read\_from\_path(MP3\_FILE)}\OperatorTok{.}\NormalTok{unwrap()}\OperatorTok{;}
    \CommentTok{// set the title}
\NormalTok{    mp3tag}\OperatorTok{.}\NormalTok{set\_title(}\StringTok{"title from mp3 file"}\NormalTok{)}\OperatorTok{;}
    \CommentTok{// we can convert it to an mp4 tag and save it to an m4a file.}
    \KeywordTok{let} \KeywordTok{mut}\NormalTok{ mp4tag }\OperatorTok{=}\NormalTok{ mp3tag}\OperatorTok{.}\NormalTok{into\_tag(}\PreprocessorTok{TagType::}\NormalTok{Mp4)}\OperatorTok{;}
\NormalTok{    mp4tag}\OperatorTok{.}\NormalTok{write\_to\_path(M4A\_FILE)}\OperatorTok{.}\NormalTok{unwrap()}\OperatorTok{;}

    \CommentTok{// reload the tag from the m4a file; this time specifying the}
    \CommentTok{// tag type (you can also use \textasciigrave{}default()\textasciigrave{})}
    \KeywordTok{let} \KeywordTok{mut}\NormalTok{ mp4tag }\OperatorTok{=} \PreprocessorTok{Tag::}\NormalTok{with\_tag\_type(}\PreprocessorTok{TagType::}\NormalTok{Mp4)}
        \OperatorTok{.}\NormalTok{read\_from\_path(M4A\_FILE)}
        \OperatorTok{.}\NormalTok{unwrap()}\OperatorTok{;}
    \CommentTok{// the tag originated from an mp3 file is successfully written}
    \CommentTok{// to an m4a file!}
    \PreprocessorTok{assert\_eq!}\NormalTok{(mp4tag}\OperatorTok{.}\NormalTok{title()}\OperatorTok{,} \ConstantTok{Some}\NormalTok{(}\StringTok{"title from mp3 file"}\NormalTok{))}\OperatorTok{;}
    \CommentTok{// multiple artists}
\NormalTok{    mp4tag}\OperatorTok{.}\NormalTok{add\_artist(}\StringTok{"artist1 of mp4"}\NormalTok{)}\OperatorTok{;}
\NormalTok{    mp4tag}\OperatorTok{.}\NormalTok{add\_artist(}\StringTok{"artist2 of mp4"}\NormalTok{)}\OperatorTok{;}
    \PreprocessorTok{assert\_eq!}\NormalTok{(}
\NormalTok{        mp4tag}\OperatorTok{.}\NormalTok{artists()}\OperatorTok{,}
        \ConstantTok{Some}\NormalTok{(}\PreprocessorTok{vec!}\NormalTok{[}\StringTok{"artist1 of mp4"}\OperatorTok{,} \StringTok{"artist2 of mp4"}\NormalTok{])}
\NormalTok{    )}\OperatorTok{;}
    \CommentTok{// convert to id3 tag, which does not support multiple artists}
\NormalTok{    mp4tag}\OperatorTok{.}\NormalTok{set\_config(}\PreprocessorTok{Config::}\KeywordTok{default}\NormalTok{()}\OperatorTok{.}\NormalTok{sep\_artist(}\StringTok{"/"}\NormalTok{))}\OperatorTok{;}
    \CommentTok{// separator is by default \textasciigrave{};\textasciigrave{} but we can customise it}
    \KeywordTok{let}\NormalTok{ mp3tag }\OperatorTok{=}\NormalTok{ mp4tag}\OperatorTok{.}\NormalTok{into\_tag(}\PreprocessorTok{TagType::}\NormalTok{Id3v2)}\OperatorTok{;}
    \PreprocessorTok{assert\_eq!}\NormalTok{(mp3tag}\OperatorTok{.}\NormalTok{artist()}\OperatorTok{,} \ConstantTok{Some}\NormalTok{(}\StringTok{"artist1 of mp4/artist2 of mp4"}\NormalTok{))}\OperatorTok{;}
\OperatorTok{\}}
\end{Highlighting}
\end{Shaded}

\hypertarget{anytag}{%
\chapter{\texorpdfstring{\texttt{AnyTag}}{AnyTag}}\label{anytag}}

The following example shows how you can create a ``generic'' \texttt{AnyTag} and convert it into a specific tag type.

\begin{Shaded}
\begin{Highlighting}[]
\KeywordTok{use} \PreprocessorTok{audiotags::}\OperatorTok{\{}\NormalTok{AnyTag}\OperatorTok{,}\NormalTok{ AudioTagEdit}\OperatorTok{,}\NormalTok{ Id3v2Tag}\OperatorTok{\};}

\KeywordTok{fn}\NormalTok{ main() }\OperatorTok{\{}
    \KeywordTok{let} \KeywordTok{mut}\NormalTok{ tag }\OperatorTok{=} \PreprocessorTok{AnyTag::}\KeywordTok{default}\NormalTok{()}\OperatorTok{;}
\NormalTok{    tag}\OperatorTok{.}\NormalTok{set\_title(}\StringTok{"foo"}\NormalTok{)}\OperatorTok{;}
\NormalTok{    tag}\OperatorTok{.}\NormalTok{set\_year(}\DecValTok{2001}\NormalTok{)}\OperatorTok{;}
    \KeywordTok{let}\NormalTok{ tag}\OperatorTok{:}\NormalTok{ Id3v2Tag }\OperatorTok{=}\NormalTok{ tag}\OperatorTok{.}\NormalTok{into()}\OperatorTok{;}
    \PreprocessorTok{assert\_eq!}\NormalTok{(tag}\OperatorTok{.}\NormalTok{year()}\OperatorTok{,} \ConstantTok{Some}\NormalTok{(}\DecValTok{2001}\NormalTok{))}\OperatorTok{;}
\NormalTok{    tag}\OperatorTok{.}\NormalTok{write\_to\_path(}\StringTok{"assets/a.mp3"}\NormalTok{)}\OperatorTok{.}\NormalTok{unwrap()}\OperatorTok{;}
\OperatorTok{\}}
\end{Highlighting}
\end{Shaded}

\hypertarget{downcast}{%
\chapter{Downcast}\label{downcast}}

The following example shows how you can downcast a \texttt{Box\textless{}dyn\ AudioTag\textgreater{}} into its ``backend'' tag type. This allows you to set the uncommon metadata supported by the corresponding backend but not by \textbf{audiotags}.

\begin{Shaded}
\begin{Highlighting}[]
\KeywordTok{use} \PreprocessorTok{audiotags::}\OperatorTok{*;}

\KeywordTok{fn}\NormalTok{ main() }\OperatorTok{\{}
    \KeywordTok{let} \KeywordTok{mut}\NormalTok{ innertag }\OperatorTok{=} \PreprocessorTok{metaflac::Tag::}\KeywordTok{default}\NormalTok{()}\OperatorTok{;}
\NormalTok{    innertag}
        \OperatorTok{.}\NormalTok{vorbis\_comments\_mut()}
        \OperatorTok{.}\NormalTok{set\_title(}\PreprocessorTok{vec!}\NormalTok{[}\StringTok{"title from metaflac::Tag"}\NormalTok{])}\OperatorTok{;}
    \KeywordTok{let}\NormalTok{ tag}\OperatorTok{:}\NormalTok{ FlacTag }\OperatorTok{=}\NormalTok{ innertag}\OperatorTok{.}\NormalTok{into()}\OperatorTok{;}
    \KeywordTok{let} \KeywordTok{mut}\NormalTok{ id3tag }\OperatorTok{=}\NormalTok{ tag}\OperatorTok{.}\NormalTok{into\_tag(}\PreprocessorTok{TagType::}\NormalTok{Id3v2)}\OperatorTok{;}
\NormalTok{    id3tag}
        \OperatorTok{.}\NormalTok{write\_to\_path(}\StringTok{"assets/a.mp3"}\NormalTok{)}
        \OperatorTok{.}\NormalTok{expect(}\StringTok{"Fail to write!"}\NormalTok{)}\OperatorTok{;}

    \KeywordTok{let}\NormalTok{ id3tag\_reload }\OperatorTok{=} \PreprocessorTok{Tag::}\KeywordTok{default}\NormalTok{()}
        \OperatorTok{.}\NormalTok{read\_from\_path(}\StringTok{"assets/a.mp3"}\NormalTok{)}
        \OperatorTok{.}\NormalTok{expect(}\StringTok{"Fail to read!"}\NormalTok{)}\OperatorTok{;}
    \PreprocessorTok{assert\_eq!}\NormalTok{(id3tag\_reload}\OperatorTok{.}\NormalTok{title()}\OperatorTok{,} \ConstantTok{Some}\NormalTok{(}\StringTok{"title from metaflac::Tag"}\NormalTok{))}\OperatorTok{;}

    \KeywordTok{let} \KeywordTok{mut}\NormalTok{ id3tag\_inner}\OperatorTok{:} \PreprocessorTok{id3::}\NormalTok{Tag }\OperatorTok{=}\NormalTok{ id3tag\_reload}\OperatorTok{.}\NormalTok{try\_into()}\OperatorTok{.}\NormalTok{unwrap()}\OperatorTok{;}
    \CommentTok{// this would fail if \textasciigrave{}id3tag\_reload\textasciigrave{} isn\textquotesingle{}t really a id3 tag.}

    \KeywordTok{let}\NormalTok{ timestamp }\OperatorTok{=} \PreprocessorTok{id3::}\NormalTok{Timestamp }\OperatorTok{\{}
\NormalTok{        year}\OperatorTok{:} \DecValTok{2013}\OperatorTok{,}
\NormalTok{        month}\OperatorTok{:} \ConstantTok{Some}\NormalTok{(}\DecValTok{2u8}\NormalTok{)}\OperatorTok{,}
\NormalTok{        day}\OperatorTok{:} \ConstantTok{Some}\NormalTok{(}\DecValTok{5u8}\NormalTok{)}\OperatorTok{,}
\NormalTok{        hour}\OperatorTok{:} \ConstantTok{Some}\NormalTok{(}\DecValTok{6u8}\NormalTok{)}\OperatorTok{,}
\NormalTok{        minute}\OperatorTok{:} \ConstantTok{None}\OperatorTok{,}
\NormalTok{        second}\OperatorTok{:} \ConstantTok{None}\OperatorTok{,}
    \OperatorTok{\};}
\NormalTok{    id3tag\_inner}\OperatorTok{.}\NormalTok{set\_date\_recorded(timestamp}\OperatorTok{.}\NormalTok{clone())}\OperatorTok{;}
\NormalTok{    id3tag\_inner}
        \OperatorTok{.}\NormalTok{write\_to\_path(}\StringTok{"assets/a.mp3"}\OperatorTok{,} \PreprocessorTok{id3::Version::}\NormalTok{Id3v24)}
        \OperatorTok{.}\NormalTok{expect(}\StringTok{"Fail to write!"}\NormalTok{)}\OperatorTok{;}

    \KeywordTok{let}\NormalTok{ id3tag\_reload }\OperatorTok{=} \PreprocessorTok{id3::Tag::}\NormalTok{read\_from\_path(}\StringTok{"assets/a.mp3"}\NormalTok{)}
        \OperatorTok{.}\NormalTok{expect(}\StringTok{"Fail to read!"}\NormalTok{)}\OperatorTok{;}
    \PreprocessorTok{assert\_eq!}\NormalTok{(id3tag\_reload}\OperatorTok{.}\NormalTok{date\_recorded()}\OperatorTok{,} \ConstantTok{Some}\NormalTok{(timestamp))}\OperatorTok{;}
\OperatorTok{\}}
\end{Highlighting}
\end{Shaded}


  \bibliography{book.bib,packages.bib}

\end{document}
